{\bfseries{Què és un arbre filogenètic?}}

En la biologia evolutiva, antropologia, lingüística i moltes altres disciplines científiques un dels problemes principals que s\textquotesingle{}aborden és la construcció d\textquotesingle{}arbres filogenètics, diagrames que representen de manera esquemàtica les relacions evolutives entre un conjunt de N entitats (en biologia evolutiva se sol parlar d\textquotesingle{}espècies, tot i que sovint no es tracti d\textquotesingle{}espècies sinó de famílies o ordres). L\textquotesingle{}arbre filogenètic es construeix en base a les similituds i diferències en les característiques físiques o genètiques de les N entitats. En un arbre filogenètic arrelat cada node amb descendents representa l\textquotesingle{}ancestre comú més recent dels seus descendents, i és usual que la longitud de les arestes / branques de l\textquotesingle{} arbre sigui proporcional a el temps transcorregut entre les entitats representades. els nodes interns són entitats hipotètiques, ja que no poden ser directament observades, només les N entitats de les fulles de l\textquotesingle{}arbre són les que coneixem. La construcció de l\textquotesingle{}arbre filogenètic que millor explica les dades observades, optimitzant un cert criteri, és un problema computacionalment difícil, en el sentit que el cost dels càlculs necessaris creix exponencialment amb N. A més molts altres problemes compliquen l\textquotesingle{}obtenció d\textquotesingle{}arbres filogenètics que reflecteixin la història evolutiva amb precisió\+: dades sobre les entitats inexactes, hibridacions, evolució convergent,. . . Per aquesta raó s\textquotesingle{}han desenvolupat nombrosos mètodes aproximats que generen arbres filogenètics de manera eficient i que s\textquotesingle{}aproximen molt bé a l\textquotesingle{}arbre òptim en la majoria de casos, però no sempre. En aquesta pràctica el nostre objectiu serà construir un programa i els mòduls necessaris per a construir l\textquotesingle{}arbre filogenètic per a un conjunt de N espècies utilitzant un d\textquotesingle{}aquests mètodes aproximats\+: el mètode conegut com W\+P\+G\+MA (Weighted pair group with arithmetic mean).

Aquest document hi trobarem les diverses classes utilitzades per poder executar bé el \mbox{\hyperlink{program_8cc}{program.\+cc}}. Hi tenim 4 classes\+: \begin{DoxyVerb}-Especie: Representa una especie amb el seu gen.

-Cjt_especies: Representa un conjunt d'especies, amb identificador i especie.

-Cluster: Representa un cluster d'un arbre binari.

-Cjt_clusters: Representa un conjunt de clusters amb identificador.
\end{DoxyVerb}


Amb aquestes classes podrem crear un conjunt d\textquotesingle{}especies, i fer diverses operacions amb elles, ja sigui eliminar especies, consultar distàncies, imprimir una taula de distàncies entre especies... Amb la classe Cjt\+\_\+cluster i cluster, podrem anar creant un arbre binari on aquest serà el resultat d\textquotesingle{}aplicar l\textquotesingle{}algorisme W\+P\+G\+MA.

{\bfseries{La descripció del programa principal d\textquotesingle{}aquesta pràctica és la següent}}\+: Donat un conjunt d\textquotesingle{}especies, hem programat un conjunt de funcions per anar modificant el conjunt i anar creant l\textquotesingle{}arbre filogenetic. 